\subsection[Mise en place: mininet et MPTCP]{Mise en place: mininet et
  MPTCP\footnote{\label{etatavannote}par M. Ly}}
\label{subsec:etatavanc:mininet-mptcp}


La compilation du noyau linux MPTCP (v0.88) dans une image VM de
mininet (v2.10) est termin�. Nous avons cr�� des paquets debian pour
faciliter l'installation sur les VM de mininet. Il est n�cessaire de
d�sinstaller openvswitch avant d'installer le nouveau noyau car
openvwitch (v0.9) n'est pas compatible avec les noyaux linux r�cents,
il faudra r�installer openvswitch ensuite.

Pour tester la r�ussite de l'installation, une topolgie deux h�tes
deux switchs a �t� utilis�. L'utilisation de MPTCP montre un d�bit par
iperf sup�rieur � la m�me exp�rience o� MPTCP a �t� d�sactiv� dans le
noyau.


\subsection[Topologies virtualis�es]{Topologies virtualis�es}
\label{subsec:etatavanc:topologgie}

\subsubsection[Topologie MPTCP vs TCP]{Topologie MPTCP vs TCP\footnotemark[1]}
\label{subsubsec:mptcpvstcp}

Nous avons reproduit la topologie o� un sous-flot MPTCP est en
concurrence avec un flux TCP \cite{pareto2013}. Il reste � �tablir les
tables de routage de chaque h�te pour pouvoir tester les performances
de MPTCP.

