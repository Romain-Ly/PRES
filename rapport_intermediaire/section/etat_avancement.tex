\subsection[Outil de coordination: git]{Outil de coordination: git}
\label{subsec:etatavanc:git}
L'�tat des scripts utilis�es par l'�quipe est mise � jour par
l'interm�diaire d'un syst�me de version utilisant git
\url{https://github.com/Romain-Ly/PRES}.

\subsection[Mise en place: mininet et MPTCP]{Mise en place: mininet et
  MPTCP\footnote{par M. Ly}}
\label{subsec:etatavanc:mininet-mptcp}


La mise en place du noyau linux MPTCP (v0.88) dans une image VM de
mininet (v2.10) est � 100~\% termin�.

Les paquets debian pour l'installation du noyau MTPCP sur les VM de
mininet se trouvent ici
(\url{https://www.dropbox.com/sh/y4ykck8rg6908ps/7V3lsV6Ggg}).

Pour tester la r�ussite de l'installation, une topologie deux h�tes
deux switchs a �t� utilis�. L'utilisation de MPTCP montre un d�bit
sup�rieur lorsque l'on compare la m�me exp�rience o� MPTCP a �t�
d�sactiv� dans le noyau.


\subsection[Topologies virtualis�es]{Topologies virtualis�es}
\label{subsec:etatavanc:topologgie}

J'ai reproduit la topologie o� MPTCP est en concurrence avec un flux
TCP \cite{pareto2013}. Il reste � �tablir les tables de routage de
chaque h�te pour pouvoir tester les performances de MPTCP.


\subsection[Mise en place: mininet et MPTCP]{Code\footnote{par M. Lam
    et M. Dubois}}
\label{subsec:etatavanc:mininet-mptcp}

Nous avons regard� les fichiers de MPTCP pour avoir une vision globale
de l'impl�mentation dans le noyau linux et essayer de d�terminer les
fichiers qui concernent l'ordonnancement des sous-flux.  Nous avons
ensuite essay� de d�terminer o� nous pouvions modifier le code afin
d'adapter l'ordonnanceur aux besoins du projet. Nous avons avanc� sur
cette phase de compr�hension du code mais il nous reste toujours �
savoir o� nous pouvons modifier le code sans rendre MPTCP non
fonctionnel ou non performant. Pour cela, il faudra tester sur des
topologies virtuelles simples et comparer les diff�rences de
performances. Bien s�r, dans les tests nous ne codons que des
ordonnanceurs idiots : ils effectueront uniquement une r�partition
�quitable des sous-flux sachant qu'ils ont tous le m�me d�bit.
