

\subsection{Topologies virtualis�es}
\label{sec:conce:topologiesvirt}

L'exp�rimentation de MPTCP \emph{in situ} peut s'av�rer difficile
lorsqu'on ne conna�t pas les chemins utilis�s par les sous-flots.

Mininet permet de cr�er un r�seau virtuel et d'utiliser un vrai noyau
en dessous (ici celui de MPTCP) dans une seule machine. Le contr�leur
du r�seau cr�� permettra d'utiliser au plein potentiel MPTCP. Il
permettra � MPTCP de contr�ler les chemins utilis�s et ainsi de
pouvoir maximer le potentiel de MPTCP.

Le noyau de MPTCP sera compil� et install� dans une machine virtuelle
contenant mininet et tournant sur ubuntu. Nous cr�erons les topologies
virtuelles gr�ce � l'API python et nous effectuerons les tests et les
mesures de performance de la m�me mani�re.

\subsection{Performances MPTCP}
\label{sec:conce:perfMPTCP}

Pour pouvoir mesurer les performances, nous allons faire varier les
propri�t�s des chemins emprunt�s pas les sous-flots de mani�re
asym�trique pour d�terminer les performances de MPTCP. Les contraintes
appliqu�es auront comme crit�res la latence (crit�re actuellement
privili�g� par l'ordonnanceur), la capacit�, le taux d'erreur, etc.



\subsection{Algorithme d'ordonnancement}
\label{sec:conc:ealgo}
L'�criture et le test de l'algorithme d'ordonnancement dans le noyau
linux peut s'av�rer une t�che difficile en peu de temps. Pour tester
la validit� de notre algorithme d'ordonnancement, nous r�fl�chisson �
effectuer un \emph{ proof of concept} en utilisant directement python
on qui utilisera des fonctions de \emph{callback} pour lancer les
fonctions du noyau n�cessaire � MPTCP. On utilisera alors UDP pour la
transmission des donn�es.
