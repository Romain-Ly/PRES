\section{Plan de d\'eveloppement}
\label{sec:plan:devt}



La première partie est de consuitre les topologies virtualisées et de
tester les performances de MTPCP en faisant varier les paramètres des
sous-flots. La seconde partie est de construire un alogrithme
d'ordonnancement répondant à des critères de sécurité.

\\Les étapes du développement suivront les points suivants:

\begin{itemize}
\item Préparation d'une machine mininet contenant MPTCP pour
  l'ensemble de l'équipe.
\item Lecture du code de MPTCP et commentaires du code.
\item Préparation de plusieurs topologies : \emph{fat tree} pour
  simuler un \emph{data center} et d'une topologie permettant de
  tester la concurrence entre MPTCP et TCP.
\item Préparation d'une bibliothèque de tests et de mesures via l'API python
\item Préparation et écriture des algorithmes d'ordonnancement
\item Compilation et tests de l'algorithme dans la machine virtuel
\end{itemize}

 





2- Préparer toute une bibliothèque de tests, mesure et autres à
appliquer sur notre réseau MPTCP // L'idéal serait d'aller un peu plus
loin que iperf quoi, faut que je voie ça avec Romain aussi 3- Préparer
les différents algorithmes d'ordonnancement // Donc Kevin et Quentin à
vous de gérer là-dessus 4- Exécuter les tests sur la Topo en alternant
les algos d'ordonnancement 5- Faire un rapport (plus tard quoi)

Donc le 3 est faisable en parallèle (en cours), il faut avoir fini le
1 (fait) pour entamer le 2 (en cours).  Quand les 1/2/3 seront finis
on passera au 4, puis au 5

Maintenant si quelqu'un qui maîtrise les diagrammes de Gantt peut
modéliser ça (je dirais pas que j'ai la flemme mais plus que j'ai pas
d'outil pour =X)

Hésitez pas à apporter vos avis =)


e rapport interm ediaire se compose :

