\subsection{arguments mininet}
\label{sec:annexe1:commentaires_argument}

\begin{verbatim}
usage: sudo mptc_khal.py [h] [--cli] --bw BW [--delay DELAY] [-n N] [-t T]
                         [--mptcp] [--pause] [--ndiffports NDIFFPORTS]
\end{verbatim}



cf argument parser annexe.

\subsubsection{ssh}
\label{sec:annexe1:ssh}

\begin{verbatim}
usage: sudo mptc_khal.py --sshd
\end{verbatim}

Cette option lance le d�mon ssh dans chacun des n\oe uds avec la
commande 
\begin{verbatim}
/usr/sbin/sshd -o UseDNS=no -u0
\end{verbatim}. 

Pour l'instant, seul la connexion � partir de root avec le premier
h�te (172.16.0.1) fonctionne. En effet la r�solution ARP �choue pour
les autres h�tes (pas de paquets ARP reply en utilisant
tcpdump).\\ 

sur le n\oe ud root
\begin{verbatim}
tcpdump -i root-eth0 arp
\end{verbatim}. 

Cependant, il reste possible de se connecter aux autres h�tes par ssh
via le premier h�te.

Pour l'instant, une m�thode \og sale \fg est utilis�e pour arr�ter les
d�mons sshd sur les h�tes.


\subsection{R�sum� des arguments pour le parseur}
\label{sec:annexe1:mininetParserargs}

Les commentaires des arguments sont disponibles dans la section
\ref{sec:annexe1:commentaires_argumentresume}.

\begin{verbatim}
usage: pyMPTCP.py [-h] [--verbose] [--tc] [--sshd] [--cli] [--csv] [--dump]
                  [--shark] [--bwm_ng] [--output OUTPUT] [--prepend PREPEND]
                  [--postpend POSTPEND] --open OPEN [--file FILE] --bw BW
                  [--delay DELAY] [-n N] [-t T] [--maxq MAXQ] [--mptcp]
                  [--pause] [--ndiffports NDIFFPORTS] [--arg1 ARG1]
                  [--arg2 ARG2] [--arg3 ARG3] [--arg4 ARG4] [--arg5 ARG5]
\end{verbatim}

\textbf{\large Description}
\vspace{0.5cm}

\begin{tabular}{lp{\linewidth - 4cm}}

-\,-mptcp & active mptcp\\

-\,-bwm& fixe la capacit� maximale de \textbf{tous} les liens\\

-\,-delay& fixe le d�lai (direction sym�trique) de \textbf{tous} les liens\\

-\,-bwm-ng& lance bwm-ng sur le cient et le serveur\\

-\,-shark & lance tcpdump sur toutes les interfaces client et serveur et �crit la sortie dans un fichier ``[h�te]-[if].pcap''\\

-\,-maxq \uline{n}& taille maxiamle de la queue au niveau des \emph{switchs}\\


-\,-open & fichier ``exp�rience'' utilis� (contenu dans le dossier ``./experiment''\\


-\,-output & dossier contenant les fichiers de sortie\\

-\,-file & nom des fichiers de sortie\\

-\,-prepend & pr�fixe pour les noms de fichiers\\

-\,-postpend & suffixe pour les noms de fichiers\\



-\,-cli &lance le mode \emph{command line interface} avant le lancement de l'exp�rience\\

-\,-csv & la sortie de la commande iperf est format�e en csv\\

-\,-delay &fixe le d�lai de tous les liens\\

-\,-pause & pause apr�s la simulation\\

-\,-sshd &lance sshd sur chaque h�te. Connexion du r�seau � l'espace de nom de la racine.\\

-\,-t \uline{n}&\emph{iperf} dur�e en seconde de l'exp�rience\\

-\,-n \uline{n}& nombre de sous-flots\\


-\,-tc & active la modification des liens via \emph{tc}\\

-\-argx & en conjonction avec -\,- tc, permet de rentrer d'autres arguments avec les fichiers \emph{experiment}\\

-\,-ndiffports \uline{n} & non utilis�\\
-\,-verbose& non utilis�\\
-\,-dump& non utilis�\\
\end{tabular}

\vspace{1cm}
exemple :
\begin{verbatim}
sudo mptpc_khal.py --cli --delay "50ms"
\end{verbatim}