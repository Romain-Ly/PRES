Analyse de la structure de l'impl�mentation :\\
   D�terminer globalement les fichiers � lire ;\\
   D�finir les headers et structures li�s � l'utilisation de mptcp.\\

Lecture et Compr�hension du code :\\
   Lecture de tous les fichiers contenus dans \$(SRC\_NOYAU)/net/mptcp ;\\
   Lecture de certains fichiers contenus dans \$(SRC\_NOYAU)/net/sched ;\\
   Lecture de tous les types/structures utilis�s ;\\
   Relecture du code apr�s avoir compris tous le types/structures.\\

D�finition des parties modifiables de l'ordonnanceur :\\
   V�rifier les correspondances entre mptcp\_output.c et mptcp\_input.c ;\\
   Voir si l'utilisation du contr�le de congestion (mptcp\_olia.c et mptcp\_coupled.c) a des effets de bord sur les sockets ;\\
   Voir les diff�rences entre les diff�rents modes du path\_manager.\\

Impl�menter / Tests d'algorithmes pour l'ordonnanceur\\ Avec l'aide de
   Matthieu Coudron, modification dans
   \$(SRC\_NOYAU)/net/mptcp/mptcp\_output.c ;\\ Tests des
   impl�mentations avec le travail de Romain LY et comparer les
   r�sultats obtenus.\\