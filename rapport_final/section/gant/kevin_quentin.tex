Analyse de la structure de l'impl�mentation :\\
	\begin{itemize}
		\item D�terminer globalement les fichiers � lire ;
		\item D�finir les headers et structures li�s � l'utilisation de mptcp.
	\end{itemize}

Lecture et Compr�hension du code :\\
	\begin{itemize}
   		\item Lecture de tous les fichiers contenus dans \$(SRC\_NOYAU)/net/mptcp ;
   		\item Lecture de certains fichiers contenus dans \$(SRC\_NOYAU)/net/sched ;
	   \item Lecture de tous les types/structures utilis�s ;
	   \item Relecture du code apr�s avoir compris tous le types/structures.
	\end{itemize}

D�finition des parties modifiables de l'ordonnanceur :\\
	\begin{itemize}
	   \item V�rifier les correspondances entre mptcp\_output.c et mptcp\_input.c ;\\
	   \item Voir si l'utilisation du contr�le de congestion (mptcp\_olia.c et mptcp\_coupled.c) a des effets de bord sur les sockets ;\\
	   \item Voir les diff�rences entre les diff�rents modes du path\_manager.\\
	\end{itemize}

Impl�mentation / Tests d'algorithmes pour l'ordonnanceur\\ 
	\begin{itemize}
		\item Avec l'aide de Matthieu Coudron, modification dans
   \$(SRC\_NOYAU)/net/mptcp/mptcp\_output.c ;
	   \item Tests des impl�mentations avec le travail de Romain LY et comparer les r�sultats obtenus.
   	\end{itemize}