
\subsubsection{Topologie}
\label{sec:annexe1:bugs:topo}

\`A la cr�ation de la topologie mininet, il est n�cessaire de donner
des noms courts aux h�tes et aux switchs.\\

Il n'est pas possible de cr�er deux liens entre les m�mes n\oe uds de
la classe ``Topo''. Pour contourner cette limitation, il faut cr�� la
topologie avec un seul lien puis dans la classe ``Mininet'', rajouter
le second lien. C'est pourquoi, dans le fichier
``\emph{pyMPTCP\_topo.py}'', il y a deux d�finitions pour la m�me
topologie: une pour la classe Topo (\emph{topo()}) et une autre pour
la classe Mininet (\emph{topo\_options(args,net)}) qui sera execut�
s�quentiellement lors de la simulation.\\


\subsubsection{mininet}
\label{sec:annexe1:bugs:mininet}

\paragraph{iperf et ping}
L'utilisation conjointe de la commande \emph{iperf} et de la commande
ping dans une simulation produit une erreur du calcul du d�lai par la
commande ping. Ce d�lai augmente exponentiellement vers environ
1000\,ms. Si on analyse les traces enregistr�s avec \emph{tcpdump},
les r�ponses aux \emph{echo request} se produisent bien plus
rapidement que ne laissent sug�rer les r�sultats de la commande
\emph{ping}. L'utilisation conjointe de ces deux applications marchent
tr�s bien entre la machine physique et la VM et il est probable que ce
probl�me soit li� � mininet.\\

\paragraph{sshd}
\label{sec:annexe1:sshd}

\begin{verbatim}
usage: sudo mptc_khal.py --sshd
\end{verbatim}

Cette option lance le d�mon ssh dans chacun des n\oe uds avec la
commande 
\begin{verbatim}
/usr/sbin/sshd -o UseDNS=no -u0
\end{verbatim}. 

Pour l'instant, seul la connexion � partir de root avec le premier
h�te (172.16.0.1) fonctionne. En effet la r�solution ARP �choue pour
les autres h�tes (pas de paquets ARP reply en utilisant
tcpdump).\\ 

sur le n\oe ud root
\begin{verbatim}
tcpdump -i root-eth0 arp
\end{verbatim}. 

Cependant, il reste possible de se connecter aux autres h�tes par ssh
via le premier h�te.


\paragraph{MSS}
Lorsque nous fixons le \emph{Maximum Segment Size} � 1460 octets, nous
obtenons ce message d'erreur:
\begin{verbatim}
  WARNING: attempt to set TCP maximum segment size to 1460, but got
  536
\end{verbatim}

Cependant, si nous analysons les paquets enregistr�s gr�ce �
\emph{tcpdump}, nous observons que le MSS n�goci� pendant le
\emph{handshake} de la connexion est bien de 1460 octets et que la
taille des paquets de donn�es est de 1428 octets.

\paragraph{iperf3}
Dans l'�tat des scripts, \emph{Iperf3} produit une erreur � la fin de
la simulation. Il est n�cessaire de nettoyer la simulation et il n'est
pas possible pour l'instant d'utiliser les scripts \emph{shell} pour
effectuer des simulations automatis�es.

\begin{verbatim}
sudo bash shMPTCP.sh clean
\end{verbatim}



