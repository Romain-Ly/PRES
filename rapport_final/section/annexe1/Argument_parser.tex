\subsection{R�sum� des arguments pour le parseur}
\label{sec:annexe1:mininetParserargs}

Les commentaires des arguments sont disponibles dans la section
\ref{sec:annexe1:commentaires_argument}.

\begin{verbatim}
usage: sudo mptc_khal.py [h] [--cli] --bw BW [--delay DELAY] [-n N] [-t T]
                         [--mptcp] [--pause] [--ndiffports NDIFFPORTS]
\end{verbatim}

\textbf{\large Description}
\vspace{0.5cm}

\begin{tabular}{lp{\linewidth - 4cm}}
-\,-cli&lance le mode \emph{command line interface} avant le lancement de l'exp�rience\\

-\,-csv & la sortie de la commande iperf est format�e en csv\\

\textbf{--delay, -D} \uline{n}&fixe le d�lai de tous les liens\\

-\,-dump & lance tcpdump sur toutes les interfaces utilisant mptcp et �crit la sortie dans un fichier de type [h�te]-[if].pcap\\

\textbf{--file, -F} <filename>&fichiers de sortie, un pour le client et un pour le serveur. Par d�faut, aucun fichier n'est cr��.\\

-\,-pause & pause apr�s la simulation\\

\textbf{--sshd}&lance sshd sur chaque h�te. Connexion du r�seau � l'espace de nom de la racine.\\


-\,-t \uline{n}&\emph{iperf} dur�e en seconde de l'exp�rience\\

-\,-tc & active la modification des liens via \emph{tc}\\


\end{tabular}

\vspace{1cm}
exemple :
\begin{verbatim}
sudo mptpc_khal.py --cli --delay "50ms"
\end{verbatim}