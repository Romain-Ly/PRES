
\section{Conclusions}
\label{sec:conclusions}

Au cours de ce projet, nous avons :

\begin{itemize}
\item compil� le noyau MPTCP dans une machine virtuelle contenant
  mininet;
\item cr�� une plateforme de tests par des scripts python et bash;
\item test� le fonctionnement de MPTCP sur une topologie simple et sur
  la topologie \emph{fat-tree};
\item test� les capacit�s et les limites de la virtualisation de
  r�seaux;
\item puis mesur� les performances de MPTCP sur des liens � d�lai
  variables;
\item enfin, nous avons �crit un nouvel algorithme d'ordonnancement;
\item et nous avons compil� le noyau avec ce nouvel algorithme
\end{itemize} (bien qu'il ne soit pas encore op�rationnel).

\section{Perspectives}
\label{sec:perspectives}
Le sujet est tr�s int�ressant et les pistes pour continuer le projet
sont nombreuses.

Dans un premier temps, il serait profitable de terminer les scripts de
base en int�grant des moyens pour mesurer la charge CPU par conteneur
(\emph{cpuacct}) ou de mesurer le RTT par la connexion TCP. Ensuite,
il serait utile de faire varier d'autres param�tres comme la
probabilit� d'erreur, la gigue sur chacun des liens pour ensuite,
effectuer des exp�riences sur des liens congestionn�s o� plusieurs
utilisateurs sont en concurrence pour la ressource. Ces tests
stringents seront utiles pour mesurer la performance des algorithmes
contenus dans le noyau et pour les comparer avec le nouvel algorithme.

Concernant l'impl�mentation de l'ordonnanceur qui est toujours en
cours de correction, terminer cet algorithme et comparer les r�sultats
des deux algorithmes serait b�n�fique : modifier, par la suite, les
crit�res d'ordonnancement selon les r�sultats obtenus avec mininet
pour l'optimiser.

Et au final, il pourrait �tre int�ressant de tester notre nouvel
algorithme en situation r�el.



\section{Remerciements}
\label{sec:remerciements}

Ce projet a �t� r�alis� dans le cadre du Master I R�seau, �
l'Universit� Pierre et Marie Curie.

Nous tenons � remercier Stefano Secci pour nous avoir permis de
r�aliser ce projet.  

Et nous remercions Yacine Bencha�b et Matthieu Coudron pour les
discussions autour du projet.
